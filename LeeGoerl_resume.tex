\documentclass[a4paper,10pt,notitlepage]{article}
\usepackage[latin1]{inputenc}
\usepackage{amsfonts}
\usepackage{amsmath}
\usepackage{amssymb}
\usepackage{amsthm}
\usepackage{hyperref}
\usepackage[american]{babel}
\usepackage[dvips]{graphicx}
\usepackage{multicol}
%\renewcommand{\baselinestretch}{3}
\oddsidemargin 0.0in
\topmargin 0.0in
\headsep 0.1in
\textwidth 6.5in
\usepackage{fancyhdr}
\pagestyle{fancy}
\lhead{\large\bfseries Lee Goerl, PhD}
%\lhead{\large\bfseries Curriculum Vitae}
\begin{document}
\begin{multicols}{2}{
\noindent %Lee Goerl\\
%Cardwell Hall \#127\\
%Manhattan, KS 66502

\noindent lgoerl@gmail.com\\
\noindent 620-786-4561\\
\\
\hspace*{-10pt}\hfill github.com/lgoerl\\
\hspace*{-10pt}\hfill linkedin.com/in/lgoerl\\
}
\end{multicols}
\vspace{-20pt}\section*{Post-Secondary Education}
    \begin{description}
    \vspace{-5pt} \item[PhD:] Kansas State University, Mathematics (December 9, 2016)
    \item[Professional Training:] The Data Incubator, Data Science (November 11, 2016)
    \end{description}

\section*{Data Science, Analytics, and Engineering Experience}

    \begin{itemize}
        \vspace{-5pt}\item \textbf{PepsiCo ROI (Feb 2021 - Feb 2022:}
            \begin{itemize}
                    \item \textbf{Saturation Curves:} Media contribution is modeled against impressions, spend, and several other factors. The media buying team uses the curves to understand ROI of their MMM plans and make informed decisions about yearly spending to find the optimal level of execution across all their channels. I interfaced between platform team and the business team to build processes around gathering periodic requirements and delivery timing, and validating/iterating curves. Additionally, worked with EU business and engineering teams to extend coverage to our EU markets tune the models to their specific needs.
                    \item \textbf{API integrations:} Built POC and productionized the first integrations on my team. Led contractors to implement additional integrations.
            \end{itemize}
        \vspace{-5pt}\item \textbf{Blue Bottle (Jul 2017 - Jan 2021):}
            \begin{itemize}
                    \item \textbf{Retail Bean and Culinary Forecasting:} \emph{V1} we built a framework in Python for training predictive ensemble SARIMAX models to generate demand forecasts for each of our food items, and a delivery/management system. Much of my work on this project was focused on improving reliability and performance, completely automating model training of forecast models, reducing lead-time for forecasting for new locations, expanding the number of forecastable items, and expanding test coverage of the codebase. Additionally, I helped design and analyze experiments to determine length of data series required, what loss functions were more performant for training our ensemble, and an in-store experiment. \emph{V2} I led a team of consulting ML engineers, providing business/process insight, building the necessary data pipelines, performance and SLA monitoring, an ensemble step, and dev ops. We rebuilt the original model and infrastructure to employ an LSTM RNN and data external to each individual sales series. Additionally, we implemented a version to forecase retail bean sales and coordinated with our production and cafe teams to implement inventory tracking.
                    \item \textbf{Analytics and BI:} Rebuilt our BI infrastructure to utilize DBT, Fivetran (in addition to our own custom integrations), and Sigma. I rebuilt our subscription eventing logic, and I implemented LTV models tying together app and eComm users supporting subscriptions, merch, retail coffee, and cafe sales.
            \end{itemize}
        \vspace{-5pt}\item \textbf{QuasiCoherent Labs:} Co-founded in 2015 to offer consulting on data products to non-profit organizations. I have consulted with clients to design and spec projects to their needs. Discussed aspects of processes to be modeled, data, and collection with their domain experts. Most recently, consulted on and provided research and graphics for the book \emph{When it Finally Happens (2019)} by Mike Pearl.
        \vspace{-5pt}\item \textbf{Peronal:} 
            \begin{itemize}
                \item \textbf{Strava based app:} Scraped Strava's website for user created cycling and running routes to implement a usable search feature. Setup a remote database for the scraped data. Deployed to Heroku an API driven front-end to interface user-based queries to find routes near a specified location.
                \item \textbf{Go playing bot:} Based on AlphaGo, written with Keras, utilizes historical professional game logs, n-step-ahead move prediction, self-play to generate data and determine model improvement, and adversarial reinforcement learning to train a best next-move generator.
                \item \textbf{misc:} Model to count/sum the number of objects in a generated image, and a GAN to generate an image with a given count/sum of objects. Modeling of algebraic functions. From-scratch implementation of an MLP.
            \end{itemize}
    \end{itemize}

\end{document}
